%!TEX root = ../main.tex
%Adding the above line, with the name of your base .tex file (in this case "thesis.tex") will allow you to compile the whole thesis even when working inside one of the chapter tex files
%%
%%
\onehalfspacing
\chapter{Conclusion}
\label{chap:conc}
The solar corona produces a number of radio phenomena, some of which well understood and others less so. The resolution at which the Sun can be observed in metric wavelengths has increased dramatically over the last 20 years both spatially and temporally. Current generation radio interferometers such as LOFAR offer sub-arcminute spatial resolution when imaging and can produce spectra with a temporal resolution of 5.12 $\mu$s. 

The use of I-LOFAR in single station mode allows us access to an additional data recording source, the Transient Buffer Boards (TBBs) which can store raw voltage data at a temporal resolution of 5 ns. I have writeen software to not only create spectra from this data but also to ``point" I-LOFAR at a position on the sky in post-processing. This allows, for the first time, the corona to be studied at the nanosecond timescale and turns I-LOFAR to one of the world's most powerful solar spectrometers. 

Two computer clusters located in the I-LOFAR control room at the Rosse Observatory, Birr, Co. Offaly, record and analyse data from I-LOFAR's Remote Station Processing boards (RSPs) and TBBs. These are the REALtime Transient Acquisition cluster (REALTA) and the TBB Acquistion Cluster (TACl). Both REALTA and TACl have been set up to be in their current operational state and are able to record data at an acceptable rate. Additional work is being undertaken to parallelise recording and data analysis to vastly increase the scientific output from both clusters. Collaboration with Griffin Foster of the Breakthrough Foundation to improve REALTA so that it can perform operations on data as it is being recorded is ongoing while colaboration with Dr. Brian Coghlan from the School of Computer Science and Statistics regarding TACl also continues.

Interferometric data obtained from the LOFAR core and remote stations on 2015-10-17 from 08:00 UTC to 14:00 UTC is currently being analysed. During this time, at 13:21 UTC, a Type III burst occurred and can be seen in LOFAR beamformed observations (Figure \ref{fig:LOFAR_spec}). Preliminary images of this burst have been made using the WSCLEAN algorithm (Figure \ref{fig:typeIIIcomp}) with an appropriate weighting scheme to be decided upon. The Type III burst shows striations in its spectrum making a direct comparison between the work of \cite{Kontar2017} using interferometric imaging rather than tied-array imaging possible. Regardless of whether this work agrees or disagrees with \cite{Kontar2017}, the result will have significant implications for radio solar physics and will either prove that there is a fundamental resolution beyond which we cannot see or, the limit does not exist and interferometric observations with longer baselines will open up further study into the solar corona for years to come. %It is hoped that this analysis will find the extent to which scattering from density inhomogeneities in the corona has on the observed source sizes at metric wavelengths.







